\part{ANStoX}
\label{anstox}

A set of AWK scripts to convert annotated source to code, tests, and documentation.


This method uses one or more \texttt{.ans}\footnote{ANnotated Source} files which are filtered through one of the AWK-language filters and mashed together to form a document. The idea is that each \texttt{.ans} file contains a number of code snippets, each with a short bit of documentation for it, and a number of test cases for it.


Examples:

\begin{verbatim}
awk -f anstocode.awk alpha.ans beta.ans >thefile.scm

awk -f anstotest.awk alpha.ans beta.ans >thefile.test

awk -f anstomd.awk dict.txt alpha.ans beta.ans >README.md

awk -f anstotex.awk dict.txt alpha.ans beta.ans >thedoc.tex

awk -f anstohtml.awk dict.txt alpha.ans beta.ans >thepage.html
\end{verbatim}

In these cases, \texttt{alpha.ans} and \texttt{beta.ans} are source files containing documentation, test tags, and code tags. The code tags are printed to the code file by \texttt{anstocode.awk}, the test tags to the test file by \texttt{anstotest.awk}, and the documentation and content of the code tags are printed in different ways to the readme document by \texttt{anstomd.awk}, to the thedoc document by \texttt{anstotex.awk}, and to the thepage document by \texttt{anstohtml.awk}. The \texttt{dict.txt} file is a special case which will be explained below.


Note that the .tex document is incomplete: it needs at least a preamble and a begin for the document, which you will have to supply. It does stick an end document to the end, at least.

\chapter{Tags}
\label{tags}
\section{Documentation text}
\label{documentation-text}
\index{Documentation text}

The original conception had an \texttt{MD} tag for documentation text. I was already writing this document when I realized that it was better to have the documentation text as free text, without any tags. I'm still getting used to it, but it's definitely a change for the better.

\begin{verbatim}

This is a short docu text.

\end{verbatim}

The documentation text is processed for styling and output by the \texttt{anstomd.awk}, \texttt{anstotex.awk}, and \texttt{anstohtml.awk} scripts.


Remember to put empty lines before and after paragraphs, otherwise the paragraphs will bleed into each other.

\section{Code}
\label{code}
\index{Code}

The \texttt{CB} tag is for code, regardless of coding language.

\begin{verbatim}
CB(
#include <stdio.h>

int main (void) {
    printf("Hello, world");
    return 0;
}
CB)
\end{verbatim}
\begin{lstlisting}
#include <stdio.h>

int main (void) {
    printf("Hello, world");
    return 0;
}
\end{lstlisting}

The contents of the \texttt{CB} tag are output without further processing by all the scripts except \texttt{anstotest.awk}. \texttt{anstomd.awk}, \texttt{anstotex.awk}, and \texttt{anstohtml} add elements around the tag's output, suitable for presenting code (i.e. triple backticks for Markdown, the \texttt{lstlisting} environment for \LaTeX{}, pre for html).

\section{Verbatim}
\label{verbatim}
\index{Verbatim}

The \texttt{VB} tag is for text that should be output with no extra processing and presented with triple backticks for Markdown, the \texttt{verbatim} environment for \LaTeX{}, and pre for html. The contents of the tag aren't output by \texttt{anstocode.awk} or \texttt{anstotest.awk}.

\section{Test cases}
\label{test-cases}
\index{Test cases}

The \texttt{TT} tag is used for test case code which will be output by \texttt{anstotest.awk} alone. The \texttt{anstotest.awk} script is geared towards the \texttt{tcltest} engine, but should be possible to convert to other test engines by editing the BEGIN and END blocks in the script.

\begin{verbatim}
TT(
::tcltest::test foobar-1.0 {try the foobar} -body {
   ...
} -output "..."
TT)
\end{verbatim}

\begin{pulledtext}
\section{Pulled text}
\label{pulled-text}
\index{Pulled text}

The \texttt{PT} tag is different. It isn't treated as a content tag. Instead, one puts it around a short range of documentation text. The point of using it is that it adds formatting around the text within it: for Markdown and html it is a preceding and a succeeding horizontal rule; for \LaTeX{} it's the beginning and end of the \texttt{pulledtext} environment. The end result is a bit of text which is marked off, like an aside.

\end{pulledtext}

\section{Lists}
\label{lists}
\index{Lists}

The \texttt{IT} and \texttt{EN} tags \emph{at the beginning of the line} renders the line as a bulleted or a numbered list item, respectively.


Example:

\begin{verbatim}
IT fee
IT fie
IT foe
\end{verbatim}
\begin{itemize}
\item  fee
\item  fie
\item  foe
\end{itemize}

The \texttt{DL} tag renders the line as a definition list item. The token LD separates the term from the definition.

\begin{verbatim}
DL an item LD with a definition.
DL another item LD with mostly the same definition.
\end{verbatim}
\begin{description}
\item[an item] with a definition.
\item[another item] with mostly the same definition.
\end{description}

The definition list is faked in Markdown, and is not guaranteed to work everywhere.

\chapter{Headings}
\label{headings}

The headings tags are the same as in html, \texttt{H1} to \texttt{H6}, only you put them at the start of the line, with the heading text following. They are translated to hash groups for Markdown, to different heading elements for \LaTeX{}, and to the obvious elements in html.

\begin{verbatim}
H4 A title
\end{verbatim}
\subsection{A title}
\label{a-title}
\index{A title}
\section{IG, EM and KB}
\label{ig-em-and-kb}
\index{IG, EM and KB}

These three beginning of the line tags:

\begin{enumerate}
\item  import a graphics element (an image)
\item  wrap the entire line in italics elements
\item  wrap the entire line in keyboard font elements
\end{enumerate}
\begin{verbatim}
IG /images/myimage.png

EM This line will be in italics
\end{verbatim}

\emph{This line will be in italics}

\section{IF, IX, NI}
\label{if-ix-ni}
\index{IF, IX, NI}

These tags are mostly useful for translation to \LaTeX{}.


\texttt{IF} imports an image, but puts it in a float and adds a caption.

\begin{verbatim}
IF /images/myimage.png Look how pretty
\end{verbatim}

\texttt{IX} makes an index entry from the text that follows it. \texttt{NI} puts a \texttt{\textbackslash noindent} in front of the line.

\chapter{Styling}
\label{styling}
\section{Appearance}
\label{appearance}
\index{Appearance}

\texttt{B}\{ \ldots  \} renders text bold. \texttt{E}\{ \ldots  \} renders text in italics. \texttt{K}\{ \ldots  \} renders text in keyboard font.

\begin{verbatim}
first I B{was}, then I E{was}, but then I K{was}
\end{verbatim}

first I \textbf{was}, then I \emph{was}, but then I \texttt{was}

\section{Links, footnotes, and references}
\label{links-footnotes-and-references}
\index{Links, footnotes, and references}

\texttt{F}\{ \ldots  \} creates a \LaTeX{} footnote, or just a parenthesized bit of text in Markdown or html. \texttt{R}\{ \ldots  \}\{ \ldots  \} inserts a page reference text in \LaTeX{} and a link in Markdown or html.

\begin{verbatim}
Lorem ipsum dolor R{sit amet}{toc-label}, consectetur adipiscing elit,
\end{verbatim}

Is rendered

\begin{verbatim}
Lorem ipsum dolor sit amet (see page \pageref{toc-label}), consectetur adipiscing elit,
\end{verbatim}

in \LaTeX{},

\begin{verbatim}
Lorem ipsum dolor [sit amet](https://github.com/hoodiecrow/ConsTcl#toc-label), consectetur adipiscing elit,
\end{verbatim}

in Markdown (clearly some personalizing is necessary), and

\begin{verbatim}
Lorem ipsum dolor <a href="https://github.com/hoodiecrow/ConsTcl#toc-label">sit amet</a>, consectetur adipiscing elit,
\end{verbatim}

in html (ditto).


\texttt{S}\{ \ldots  \}\{ \ldots  \} is the same as \texttt{R}, but adds elements for keyboard font around the anchor.


\texttt{L}\{ \ldots  \}\{ \ldots  \} is like \texttt{R}, but for an external link.

\begin{verbatim}
Lorem ipsum dolor L{sit amet}{http://site/dir/index.html}, consectetur adipiscing elit,
\end{verbatim}

Is rendered

\begin{verbatim}
Lorem ipsum dolor sit amet\footnote{See \texttt{http://site/dir/index.html}}, consectetur adipiscing elit,
\end{verbatim}

in \LaTeX{},

\begin{verbatim}
Lorem ipsum dolor [sit amet](http://site/dir/index.html), consectetur adipiscing elit,
\end{verbatim}

in Markdown, and

\begin{verbatim}
Lorem ipsum dolor <a href="http://site/dir/index.html">sit amet</a>, consectetur adipiscing elit,
\end{verbatim}

in html.


\texttt{W}\{ \ldots  \}\{ \ldots  \} is like \texttt{L}, but for a link to Wikipedia. The contents of the second capture field is supposed to be the part of the URL after \texttt{https://en.wikipedia.org/wiki/}. Example:

\begin{verbatim}
W{Ann Arbor, Michigan}{Ann_Arbor,_Michigan}
\end{verbatim}

Ann Arbor, Michigan\footnote{See \texttt{https://en.wikipedia.org/wiki/Ann\_Arbor,\_Michigan}}

\section{dict.txt}
\label{dicttxt}
\index{dict.txt}

Is a file containing a dictionary that I use to construct prototype tables for procedures in my code. You will probably not need or want it, so just add a \texttt{dict.txt} containing the text "foo -> bar" or something like that (or just take the \texttt{dict.txt} from the repository). With some editing of the AWK scripts, the use for it can be removed. I still need it, so I'm not going to remove it myself.

\end{document}
